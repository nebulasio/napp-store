\section{概要}
从2008年诞生,比特币逐渐受到越来越多的人的认可,这也证明了``去中心化''这一想法的可行性。``去中心化'',很大程度上解决了现代社会所需的信任问题,通过技术提供了廉价的构建信任的方式。

以太坊更进一步的把``去中心化''这一思想从数字货币领域扩展到了任意协议,引入了图灵完备的智能合约。截止目前,智能合约不管在数量还是在多样性方面都证明了其巨大的潜力。
更具体来说,去中心化应用有以下两个个特性,1,应用行为的透明性,即,用户能够看到并理解去中心化应用的行为,去中心化应用不能通过宣传等手段夸大、捏造或歪曲应用的行为;
2,应用相关的数据的透明性,即,用户能够理解去中心化应用所使用的数据,并不会造成额外的信息泄漏,这对于用户的隐私保护而言是十分重要的。

然而,也应该看到,目前的智能合约依然存在诸多问题,例如,用户使用智能合约必须配合相应的钱包或插件,智能合约提供的功能依旧十分简单等。
究其原因,我们认为智能合约很难应对用户日益增加的、对``去中心化''应用的的需求,这是因为:
\begin{itemize}
\item \textbf{智能合约所涉及的操作相对简单} --- 这是因为智能合约的所有结果都需要上链,而目前区块链提供的TPS还十分有限,因此,难以在智能合约中进行相对复杂的操作;
\item \textbf{智能合约的交互相对简单} --- 目前,用户与智能合约的交互需要通过RPC的方式,这极大的限制了更加复杂多样的交互需求,例如,用户难以进行图形化的操作
\end{itemize}

为此,我们提出NApp Store(Nebulas decentralized Application Store),试图提供一个更好的去中心化应用分发平台,这既包括适用于用户的客户端,NApp Store Client,也包括用户开发者的开发套装,NApp dev tools。

注意,我们并不使用DApp这个术语,而是使用NApp。从概念上来说,二者皆指去中心化应用,但是,DApp一定程度上被广泛的认为是完成一定功能的智能合约及其上的前端,而本文描述的NApp则不通过智能合约或前端实现。因此,本文使用NApp,以做区分。

本文首先描述了去中心应用平台NApp Store的整体架构,然后,更进一步的介绍了客户端、开发者工具及一些相关问题的讨论。

