\section{系统架构}

NReader可以有多个实例,不同的实例可以视为不同的出版商或媒体,例如系统中可以存在一个用于发布新闻的NReader实例、一个用于发布网络小说的NReader实例、
一个用于某垂直领域咨询的NReader实例。多个实例之间是相互独立的,也就是说一个实例的NReader Client不能查看另一个实例的NReader for Author发布的内容。

每个实例包含四个部分:
\begin{itemize}
\item \textbf{NReader for Author} --- 供作者使用的客户端,这是一个NApp Store之上的NApp;
\item \textbf{Bookkeeper} --- 用于记录所有书目及元信息的智能合约;
\item \textbf{NReader Client} --- 特指读者使用的客户端,这是一个NApp Store之上的NApp;
\item \textbf{Payment Checker} --- 用于完成内容交付及支付确认的智能合约。 
\end{itemize}
\noindent 注意,对于NReader Client及NReader for Author这两个NApp,需要通过NApp Store Client安装。

不同的NReader实例由不同的主体提交、部署,因此,在NApp Store Client中会存在多个不同的NReader Client及NReader for Author,同样地,在链上则存在多个不同的Bookkeeper及Payment Checker。

NReader适用于发布持续更新的小说、专栏、新闻、咨询等,因此,NReader按照一定方式组织内容,并使得该组织方式具有一定的通用性。
NReader使用Bookkeeper组织最上层的\textbf{阅读单元},一个阅读单元可以表示一本小说、一个报刊、一个新闻门类或者一个专栏。Bookkeeper中记录多个阅读单元。
一个阅读单元包含多个\textbf{内容单元},一个内容单元可以理解为小说中的一个章节、报刊中的某一期、新闻门类中的某篇新闻稿或者专栏中的某篇文章。
内容单元也是NReader中最小的付费购买单元。

NReader的交互有几个不同的环节,包括作者创建阅读单元、新增内容单元,读者购买内容单元,作者完成内容交付几个环节。更多的环节,如作者或读者查看阅读单元及内容单元列表,是简单且直观的,因此,本文不再赘述。

\subsection{作者创建阅读单元、新增内容单元}
作者创建阅读单元、新增阅读单元都是通过NReader for Author与相关的智能合约交互。

\paragraph{创建阅读单元}
创建阅读单元是通过NReader for Author向Bookkeeper提交如下的信息
\[
(n, d, a, h)
\]
\noindent 其中\(n\)为阅读单元的名字,长度不超过15个字,\(d\)为阅读单元的简介,长度不超过140,\(a\)为提交者的地址,\(h\)为hash值,其计算方式为
\[h = H(n, d, a)\]
\noindent 其中\(H\)为hash函数。

\paragraph{新增内容单元}
向阅读单元\(i\)新增内容单元\(j\)是通过NReader for Author向Payment Checker提交如下的信息
\[(h_i, x_j, t_j, b_j, p_j, g_j )\]
\noindent 其中\(h_i\)为阅读单元\(i\)在Bookkeeper中对应的hash,\(x_j\)是该内容单元在阅读单元中的序号,按从小到大排列,\(t_j\)是该内容单元的题目,长度不超过15,\(b_j\)为该内容单元的简介,不超过50,\(p_j\)是该内容单元的价格,
\(g_j\)则为该内容单元的hash。

注意,新增阅读单元并不会将内容单元本身的明文信息提交到区块链上,因此,并不会泄露内容本身。

\subsection{购买内容单元}
购买内容单元需要通过NReader Client操作。NReader Client会通过NApp Store Client同步的区块信息得到最新的阅读单元及内容单元,并展示给用户。

用户在选择指定的内容单元进行购买之后,需要支付一定的NAS给Payment Checker,如\reffig{fig:buy}中\circled{1}所示。

如果支付的NAS低于作者指定的定价,则支付失败。如果在一定时间之后,内容仍未支付,读者可以要求退款。

\subsection{内容交付}
完成内容交付的流程是由NReader for Author自动完成的,并不需要作者参与,但是需要NReader for Author在线。

NReader for Author通过同步区块信息了解到当前相应内容单元的支付情况,如\reffig{fig:buy}中\circled{2}所示,此时,NReader for Author会从Payment Checker中得到以下信息,
\[(r, g)\]
\noindent 其中,\(r\)为所购买的读者的公钥(地址?),\(g\)为读者所购买的内容单元的hash。此时,NReader for Author会在本地查找到\(g\)所对应的
内容明文,并对明文使用\(r\)进行加密,得到加密的内容单元\(m^{\prime}\),并将其发送到Payment Checker,如\reffig{fig:buy}中\circled{3}所示,即,发送内容如下
\[(r, g, m^{\prime})\]
Payment Checker在收到相应的内容之后,需要验证\(m^{\prime}\)对应的hash正是\(g\),此处需要使用{\color{red} 零知识证明(需要进一步验证该技术细节)},如\reffig{fig:buy}中\circled{4}所示。在验证通过后,Payment Checker将相应的NAS发放给作者,并将相应的密文发送给读者,如\reffig{fig:buy}中\circled{5}所示。

\begin{figure}
\centering
\begin{tikzpicture}
\tikzset{
  base/.style={draw, on grid, align=center, minimum height=2.5ex},
  pn/.style={base, rectangle},
}

\node [pn] (a) at (0, 0) {Payment Checker};
\node [pn] (b) at (-2, -2) {NReader for Author};
\node [pn] (c) at (2, -2) {NReader Client};

\draw[->, >=stealth] (c) to [out=80, in=340]  node [midway, right] {
\circled{1}}  (a);
\draw[->, dashed, >=stealth] (a) to [out=200, in=110] node [midway, left]{\circled{2}} (b);

\draw[->, >=stealth] (b) to [out=70, in=230] node [midway, left]{\circled{3}} (a);

\node at (a.north) {\circled{4}};

\draw[->, dashed, >=stealth] (a) to [out=260, in=40] node [midway, right]{\circled{5}} (b);
\draw[->, dashed, >=stealth] (a) to [out=300, in=110] node [midway, left]{\circled{5}} (c);

\node [anchor=west] at (5, 0) {\circled{1}:支付};
\node [anchor=west] at (5, -0.5) {\circled{2}:获取支付信息};
\node [anchor=west] at (5, -1) {\circled{3}:提交密文};
\node [anchor=west] at (5, -1.5) {\circled{4}:验证密文是否正确};
\node [anchor=west] at (5, -2) {\circled{5}:完成最终支付及内容交付};
\end{tikzpicture}
\caption{内容购买及交付流程}
\label{fig:buy}
\end{figure}