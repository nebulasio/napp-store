\section{概要}
阅读内容的售卖从传统的出版行业,到数字内容的分发;从直接售卖书籍获利,到自媒体通过流量及广告获利,经历了诸多变革。然而,在这些变革中,内容的创作者的获利方式并没有发生根本性的变化。

阅读内容的主要获利方依然是中心化的组织或平台。这些中心化的组织或平台可能是出版商、新闻咨询类的门户网站、文学平台(例如起点中文网)或流量平台(例如微信)。
通常,在与中心化的组织或平台的谈判中,内容的创作者处于较为弱势的地位,因此,其利益很大程度上难以保证。


我们希望使用区块链技术、为阅读内容的创作者提供一个能够直接触达用户的阅读平台NReader,在这样的平台上,内容创作者能够直接发布其内容、为其内容定价,
而读者则能够直接使用代币兑换相应的内容,而不经过任何的第三方。

虽然使用了区块链技术,内容创作者发布的内容并不会泄露给大众,而是使用了读者的公钥进行加密,只有付费的读者能够阅读相应的内容,最大限度的保护了内容创作者的权益。

这样的阅读平台是开放的,任何组织或个人都能够创建专属的NReader,并进行垂直领域或有针对性的整合,并收取一定的服务费用,这样的服务费用是公开、透明的。

由于NReader在支付、创作者权益保障、开放性方面的特性,我们希望NReader能够为阅读内容的创作、流通提供一个更加健康、可持续发展的环境。

